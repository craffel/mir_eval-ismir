% -----------------------------------------------
% Template for ISMIR 2014
% (based on earlier ISMIR templates)
% -----------------------------------------------

\documentclass{article}
\usepackage{ismir2014,amsmath,cite}
\usepackage{graphicx}

% Title.
% ------
\title{PAPER TEMPLATE FOR ISMIR 2014}

% Single address
% To use with only one author or several with the same address
% ---------------
%\oneauthor
% {Names should be omitted for double-blind reviewing}
% {Affiliations should be omitted for double-blind reviewing}

% Two addresses
% --------------
%\twoauthors
%  {First author} {School \\ Department}
%  {Second author} {Company \\ Address}

% Three addresses
% --------------
\threeauthors
  {First author} {Affiliation1 \\ {\tt author1@ismir.edu}}
  {Second author} {\bf Retain these fake authors in\\\bf submission to preserve the formatting}
  {Third author} {Affiliation3 \\ {\tt author3@ismir.edu}}

% Four addresses
% --------------
%\fourauthors
%  {First author} {Affiliation1 \\ {\tt author1@ismir.edu}}
%  {Second author}{Affiliation2 \\ {\tt author2@ismir.edu}}
%  {Third author} {Affiliation3 \\ {\tt author3@ismir.edu}}
%  {Fourth author} {Affiliation4 \\ {\tt author4@ismir.edu}}

\begin{document}
%
\maketitle
%
\begin{abstract}
The abstract should be placed at the top left column and should contain about 150-200 words.
\end{abstract}
%
\section{Introduction}\label{sec:introduction}

This template includes all the information about formatting manuscripts for the ISMIR 2014.
Please follow these guidelines to give the final proceedings a uniform look.
If you have any questions, please contact the Conference Management.
This template can be downloaded from the ISMIR 2014 web site (http://ismir2014.ismir.net).

\section{Page Size}\label{sec:page_size}

The proceedings will be printed on
 \underline{portrait A4-size paper} \underline{(21.0cm x 29.7cm)}.
All material on each page should fit within a rectangle of 17.2cm x 25.2cm,
centered on the page, beginning 2.0cm
from the top of the page and ending with 2.5cm from the bottom.
The left and right margins should be 1.9cm.
The text should be in two 8.2cm columns with a 0.8cm gutter.
All text must be in a two-column format.
Text must be fully justified.

\section{Typeset Text}\label{sec:typeset_text}

\subsection{Normal or Body Text}\label{subsec:body}

Please use a 10pt (point) Times font. Sans-serif or non-proportional fonts
can be used only for special purposes, such as distinguishing source code text.

The first paragraph in each section should not be indented, but all other paragraphs should be.

\subsection{Title and Authors}

The title is 14pt Times, bold, caps, upper case, centered.
Authors' names are omitted when submitting for double-blind reviewing.
The following is for making a camera-ready version.
Authors' names are centered.
The lead author's name is to be listed first (left-most), and the co-authors' names after.
If the addresses for all authors are the same, include the address only once, centered.
If the authors have different addresses, put the addresses, evenly spaced, under each authors' name.

\subsection{First Page Copyright Notice}

Please include the copyright notice exactly as it appears here in the lower left-hand corner of the page.
It is set in 8pt Times.

\subsection{Page Numbering, Headers and Footers}

Do not include headers, footers or page numbers in your submission.
These will be added when the publications are assembled.

\section{First Level Headings}

First level headings are in Times 10pt bold,
centered with 1 line of space above the section head, and 1/2 space below it.
For a section header immediately followed by a subsection header, the space should be merged.

\subsection{Chord Recognition}

% relevant papers
% - pauwels & peeters
% - matthais
% - harte
% - mirex 
% - Utrecht agreement

Despite being one of the oldest MIREX tasks, evaluation methodology and metrics for automatic chord recognition is an ongoing topic of discussion.
%For example, ISMIR 2010 saw the fondly dubbed ``Utrecht Agreement on Chord Evaluation''\footnote{http://www.music-ir.org/mirex/wiki/The\_Utrecht\_Agreement\_on\_Chord\_Evaluation}, in which interested researchers met to discuss sane ways of quantifying the performance of automatic chord recognition systems.
Several recent articles address issues and concerns with vocabularies, comparison semantics, and other lexicographical challenges unique to chord recognition \cite{}.
Ultimately, the source of this difficulty stems from the inherent subjectivity in ``spelling'' a chord name and the level of detail a human observer can provide in a reference annotation \cite{McVicar}.
As a result, a consensus has yet to be reached regarding the single best approach to comparing two sequences of chord labels, and instead are often compared over a set of rules, e.g Major-Minor, Sevenths, with or without inversions, and so on.

Thanks to the previous efforts of Harte \cite{}, text representations of chord labels adhere to a standardized format, consisting of a root, quality, extensions, and a bass note; of these, only the root is strictly required.
However, in order to efficiently compare chords in a variety of different ways, it is helpful to first translate a given chord label $\mathcal{C}$ into a numerical representation, shown in Figure \ref{}.
In this example, a $G:7(9)/5$ is mapped to split into 4 pieces of information: one, the root is mapped to an absolute pitch class $\mathcal{R}$, in $[0, 11]$, where $C\to0$, $C\sharp/D\flat\to1$, etc; two, the quality is mapped to a root-invariant 12-dimensional bit vector $\mathcal{Q}$ by setting the scale degrees of the quality; three, any extensions are applied (via addition or omission) to the quality bit vector as scale degrees in a single octave, resulting in pitch vector $\mathcal{P}$; and four, the bass interval (5) is translated to the relative scale degree in semitones $\mathcal{B}$.
Note that the add-9 is rolled into a single octave as an add-2.
This is a matter of convenience as extended chords (9's, 11's or 13's) are traditionally resolved to a single-octave equivalent, but the bit-vector representation could be easily expanded to represent such information.

Having gone through this bit of effort, it is now straightforward to compare chords along the five rules used in MIREX 2013:
\begin{enumerate}
\item Root:
	\begin{enumerate}
	\item $\mathcal{R}_{est} == \mathcal{R}_{ref}$
	\item $\forall \mathcal{Q}_{ref}$
	\end{enumerate}
\item Major-Minor: Rule 1.a, plus
	\begin{enumerate}
	\item $\mathcal{Q}_{est} == \mathcal{Q}_{ref}$
	\item $\mathcal{Q}_{ref} \in \{Maj, min\}$ 
	\end{enumerate}
\item Major-Minor w/Inversions: Rule 2, plus
	\begin{enumerate}
	\item $\mathcal{B}_{ref} \in \mathcal{Q}_{ref}$
	\end{enumerate}
\item Sevenths: Rule 1.1, plus
	\begin{enumerate}
	\item $\mathcal{Q}_{est} == \mathcal{Q}_{ref} $
	\item $ \mathcal{Q}_{ref} \in \{Maj, min, Maj7, min7, 7\}$ 
	\end{enumerate}
\item Sevenths w/Inversions: Rule 4, plus
	\begin{enumerate}
	\item $\mathcal{B}_{ref} \in \mathcal{Q}_{ref}$ 
	\end{enumerate}
\end{enumerate}

Following recent trends in MIREX, an overall score is computed by weighting each comparison by the duration of its interval, over all intervals; stated another way, this is the piecewise continuous-time integral of the intersection of two chord sequences, $(\mathbf{C}_{ref}, \mathbf{C}_{est})$, expressed as follows:

\begin{equation}
S(\mathbf{C}_{ref}, \mathbf{C}_{est}) = \frac{1}{T}\int_{t=0}^{T} \mathcal{C}_{ref}(t) == \mathcal{C}_{est}(t)
\end{equation}

\noindent Here, this is achieved here by forming the union of the boundaries in each sequence, and summing the time intervals of the correct ranges. Note that equivalence is subject to one of the rules defined previously. 

Finally, the total score over a set of $N$ items is given by a discrete summation, where the importance of each score, $S_n$, is weighted by the duration, $T_n$, of each annotation:

\begin{equation}
S_{total} = \frac{\sum_{n=0}^{N} T_n*S_n}{\sum_{n=0}^{N} T_n}
\end{equation}
 

\subsubsection{Third and Further Level Headings}

Third level headings are in Times 10pt italic, flush left,
with 1/2 line of space above the section head, and 1/2 space below it.
The first letter of each significant word is capitalized.

Using more than three levels of headings is highly discouraged.

\section{Footnotes and Figures}

\subsection{Footnotes}

Indicate footnotes with a number in the text.\footnote{This is a footnote.}
Use 8pt type for footnotes. Place the footnotes at the bottom of the page on which they appear.
Precede the footnote with a 0.5pt horizontal rule.

\subsection{Figures, Tables and Captions}

All artwork must be centered, neat, clean, and legible.
All lines should be very dark for purposes of reproduction and art work should not be hand-drawn.
The proceedings are not in color, and therefore all figures must make sense in black-and-white form.
Figure and table numbers and captions always appear below the figure.
Leave 1 line space between the figure or table and the caption.
Each figure or table is numbered consecutively. Captions should be Times 10pt.
Place tables/figures in text as close to the reference as possible.
References to tables and figures should be capitalized, for example:
see \figref{fig:example} and \tabref{tab:example}.
Figures and tables may extend across both columns to a maximum width of 17.2cm.

\begin{table}
 \begin{center}
 \begin{tabular}{|l|l|}
  \hline
  String value & Numeric value \\
  \hline
  Hello ISMIR  & 2014 \\
  \hline
 \end{tabular}
\end{center}
 \caption{Table captions should be placed below the table.}
 \label{tab:example}
\end{table}

%\begin{figure}
% \centerline{\framebox{
% \includegraphics[width=\columnwidth]{figure.png}}}
% \caption{Figure captions should be placed below the figure.}
% \label{fig:example}
%\end{figure}

\section{Equations}

Equations should be placed on separated lines and numbered.
The number should be on the right side, in parentheses.

\begin{equation}
E=mc^{2}
\end{equation}

\section{Citations}

All bibliographical references should be listed at the end,
inside a section named ``REFERENCES,'' numbered and in alphabetical order.
Also, all references listed should be cited in the text.
When referring to a document, type the numbering square brackets
\cite{Author:00} or \cite{Author:00,Someone:10,Someone:04}.

\begin{thebibliography}{citations}

\bibitem {Author:00}
E. Author:
``The Title of the Conference Paper,''
{\it Proceedings of the International Symposium
on Music Information Retrieval}, pp.~000--111, 2000.

\bibitem{Someone:10}
A. Someone, B. Someone, and C. Someone:
``The Title of the Journal Paper,''
{\it Journal of New Music Research},
Vol.~A, No.~B, pp.~111--222, 2010.

\bibitem{Someone:04} X. Someone and Y. Someone: {\it Title of the Book},
    Editorial Acme, Porto, 2012.

\end{thebibliography}

%\bibliography{ismir2014template}

\end{document}
